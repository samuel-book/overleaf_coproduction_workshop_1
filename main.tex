%%%%%%%%%%%%%%%%%%%%%%%%%%%%%%%%%%%%%%%%%%%%%%%%%%%%%%%%%%%%%%%

% Set up document

\documentclass{beamer}
\usecolortheme{whale}
\setbeamersize{text margin left=5mm,text margin right=5mm}

% Used to create a section slide between section
\AtBeginSection[]{
  \begin{frame}
  \vfill
  \centering
  \begin{beamercolorbox}[sep=8pt,center,shadow=true,rounded=true]{title}
    \usebeamerfont{title}\insertsectionhead\par%
  \end{beamercolorbox}
  \vfill
  \end{frame}
}

% Remove default navigation symbols and add just  page number
\setbeamertemplate{navigation symbols}{} % Clear default navigation
\addtobeamertemplate{navigation symbols}{}{%
    \usebeamerfont{footline}%
    \usebeamercolor[fg]{footline}%
    \hspace{1em}%
    \insertframenumber/\inserttotalframenumber
}


%%%%%%%%%%%%%%%%%%%%%%%%%%%%%%%%%%%%%%%%%%%%%%%%%%%%%%%%%%%%%%%

% Title page

\title{Stroke Audit Machine Learning (SAMueL) \\ Workshop 1}
\subtitle{Investigating variation in clinical decision-making with explainable AI}


\author{Kerry Pearn\inst{1}, Michael Allen\inst{1}, Keira Pratt-Boyden\inst{1}, Martin James\inst{1,2} }
\institute{\inst{1} University of Exeter Medical School \inst{2} Royal Devon University Healthcare NHS Foundation Trust}

%\institute{Overleaf}
\date{November 2022}


\begin{document}

\frame{\titlepage}

%%%%%%%%%%%%%%%%%%%%%%%%%%%%%%%%%%%%%%%%%%%%%%%%%%%%%%%%%%%%%%%

\begin{frame}
\frametitle{Breaking down the emergency stroke pathway into key steps}
\begin{center}
\includegraphics[width=1.0\textwidth]{./images/pathway}
\end{center}
We can model key changes to pathway:
\begin{itemize}
    \item What if the pathway were faster?
    \item What if hospital determined the stroke onset time in more patients?
    \item What if clinical decision-making was like that of \emph{benchmark} hospitals? (Predict what treatment a patient would receive at other hospitals).
\end{itemize}
\end{frame}

%%%%%%%%%%%%%%%%%%%%%%%%%%%%%%%%%%%%%%%%%%%%%%%%%%%%%%%%%%%%%%%

\begin{frame}
\frametitle{Machine learning overview}
\begin{center}
\includegraphics[width=0.95\textwidth]{./images/ml_model_high_level}
\end{center}

We accessed 240,000 emergency stroke admissions in England and Wales over three years.
\end{frame}


%%%%%%%%%%%%%%%%%%%%%%%%%%%%%%%%%%%%%%%%%%%%%%%%%%%%%%%%%%%%%%%

\begin{frame}
\frametitle{Explaining model predictions with SHAP values}

SHAP values show the influence of features (even for \emph{`black box'} models).

\begin{center}
\includegraphics[width=0.85\textwidth]{./images/xgb_waterfall_low_probability.jpg}
\end{center}
\end{frame}


%%%%%%%%%%%%%%%%%%%%%%%%%%%%%%%%%%%%%%%%%%%%%%%%%%%%%%%%%%%%%%%

\begin{frame}
\frametitle{What drives use of thrombolysis across all hospitals?}

\footnotesize{Note: SHAP values here are \emph{log odds}. Each step-change in value of \textpm 1 changes the chances of receiving thrombolysis about 3-fold.}

\begin{center}
\includegraphics[width=0.83\textwidth]{./images/xgb_thrombolysis_shap_violin.jpg}
\end{center}
\end{frame}

%%%%%%%%%%%%%%%%%%%%%%%%%%%%%%%%%%%%%%%%%%%%%%%%%%%%%%%%%%%%%%%

\begin{frame}
\frametitle{When will low thrombolysing units not use thrombolysis when higher thrombolysing would?}

\footnotesize{Here, a high SHAP shows when a low-thrombolysing unit will reject use of thrombolysis when a higher thrombolysing hospital would use thrombolysis.}

\begin{center}
\includegraphics[width=0.80\textwidth]{./images/xgb_predicting_difference_shap_violin.jpg}
\end{center}
\end{frame}

%%%%%%%%%%%%%%%%%%%%%%%%%%%%%%%%%%%%%%%%%%%%%%%%%%%%%%%%%%%%%%%
\begin{frame}
\frametitle{Key findings}
General observations about thrombolysis use: The chance of receiving thrombolysis is increased by:
\emph{
\begin{itemize}
    \item Shorter arrival-to-scan times
    \item Mid-level stroke severity
    \item Precise onset time
    \item Lower pre-stroke disability
\end{itemize}
}

\vspace{5mm}

Lower thrombolysing units are particularly less likely to give thrombolysis to patients with:
\emph{
\begin{itemize}
    \item Low or high stroke severity
    \item Higher pre-stroke disability
    \item Estimated onset time
\end{itemize}
}

\end{frame}

%%%%%%%%%%%%%%%%%%%%%%%%%%%%%%%%%%%%%%%%%%%%%%%%%%%%%%%%%%%%%%%
\begin{frame}
\frametitle{Possible questions for discussion}

\begin{itemize}
    \item Does anything surprise you here?
    \item What are your attitudes to mild strokes and patients with estimated stroke-onset times?
    \item What might you do with the knowledge that different hospitals have different attitudes to mild strokes and patients with estimated stroke-onset times?
    \item Why do you think pre-stroke disability appears influential in thrombolysis decision-making?
\end{itemize}

\end{frame}

\end{document}




